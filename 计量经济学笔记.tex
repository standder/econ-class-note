\documentclass[a4paper,UTF8]{article}
\usepackage{ctex}
\title{计量经济学课堂笔记}



\begin{document}
\maketitle
\newpage
\tableofcontents
\newpage

\part{一元线性回归}
一元线性回归总共分为四个步骤,分别是模型建立,模型求解,模型验证,模型应用。
\section{模型建立}
一元线性回归的模型总共有四种表达形式:
\begin{equation}
E(X_i|Y_i) = \hat{\beta_0} + \hat{\beta_1}X_1 + \mu_i
\end{equation}

\begin{equation}
Y_i = \hat{\beta_0} + \hat{\beta_1}X_i+\mu_i
\end{equation}

\begin{equation}
\hat{Y_i} = \hat{\beta_0} + \hat{\beta_1}x_i
\end{equation}

\begin{equation}
{Y_i} = \hat{\beta_0} + \hat{\beta_1}x_i + \mu_i
\end{equation}


其中公式(1)为总体回归函数的条件均值形式,(2)为总体回归函数的真实值形式,(3)为样本回归函数的预测值形式,(4)为样本回归函数的真实值形式。


\section{模型求解}
  一元线性回归模型求解使用ols(普通最小二乘法)进行求解,求解过程不用理会。ols核心思想为残差平方和最小即$min\sum_{n=1}^{i}\mu_i $。
  通过ols求解出$\beta_0$与$\beta_1$等参数,得出线性回归方程。
\section{模型验证}
\subsection{拟合优度检验}
拟合优度检验量$R^2 = \frac{ESS}{TSS}$。$R^2$取值范围为[0,1]。$R^2$越大解释变量$X_i$对观测量$Y_i$解释力度越强。当$R^2$高于一定阈值时说明,解释变量$x_i$对观测量$y_i$具有较强的解释力度。 
\subsection{回归系数假设检验}
回归系数检验使用t统计量,其中$t = \frac{\hat{\beta_i}}{se\beta_i}$。计算出t统计量后与临界值相比较,若t统计量高于临界值则说明该解释变量$x_i$对于观测量$y_i$具有显著性影响。 
\section{模型应用}
当通过ols计算出模型$y_i = \hat{\beta_0}+\hat{\beta_1}x_i + \mu_i$,并通过了拟合优度检验与回归系数检验后。可以对模型进行应用。
此时每当解释变量$x_i$增加一个单位时,观测量$y_i$平均增加$\beta_1$个单位(千万不要漏掉平均!!!)
\section{补充知识}

\subsection{三种平方和(总平方和、回归平方和、残差平方和)} 
\begin{equation}
TSS = ESS + RSS
\end{equation}
\begin{equation}
\sum_{i=1}^{n}(Y_i-\overline{Y})^{2} = \sum{(Y_i-\overline{Y}})^{2} + \sum(\hat{Y}-\overline{Y})^{2}
\end{equation}
公式(5)与公式(6)是完全等价的。
\subsection{三种平方和的对应自由度} 

\begin{table}[h]
\caption{三种平方和的对应自由度}
\centering
\begin{tabular}{|c|c|c|}
\hline
TSS & 总回归平方和 & n-1 \\
\hline
RSS & 残差平方和 & n-k \\
\hline
ESS & 回归平方和 & k-1 \\
\hline
\end{tabular}
\end{table}

\subsection{一元线性回归的古典假定}
5个古典假定全部都是对于随机扰动项$\mu_i$而言的

1. 0均值
\begin{equation}
E(\mu_i|X_i) = 0 = E(Y_i|X_i)
\end{equation}

2. 同方差
\begin{equation}
Var(\mu_i|X_i) = \sigma^2
\end{equation}

3. 随机扰动项$\mu_i$之间逐次不相关
\begin{equation}
Cov(\mu_i,\mu_j) = 0
\end{equation}

4. 随机扰动项$\mu_i$与解释变脸$X_i$之间逐次不相关
\begin{equation}
Cov(\mu_i,X_i) = 0
\end{equation}

5. 正态性假定
\begin{equation}
\mu_i \sim N(0,\sigma^2)
\end{equation}

\subsection{最佳线性无偏估计量}
  BLUE( the best linear unbiased estimator )即最佳线性无偏估计量,应满足以下3点特性:
  
1. 无偏性:
满足$E(\mu_i) = 0$的古典假定

2. 有效性:
满足$Var(\mu_i) = \sigma^2$的古典假定

3. 线性性:
满足$Cov(\mu_i,\mu_j) = 0$的古典假定


\part{多元线性回归}

\section{模型设立}
多元线性模型
\section{模型求解}
\section{模型检验}
\section{模型应用}


\part{多重共线性}
\part{异方差}
\part{自相关}

\end{document}